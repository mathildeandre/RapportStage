\subsubsection{Les API utilisant les FlowDetails}
Dans la plateforme Amex, il y a diff�rentes services  qui utilisent les flowdetails. Je vais d�crire le fonctionnement de chacun d'eux. \\

Le service d'audit a pour but de tracer tous les messages transitant par Amex. L'audit se fait dans des tables en base de donn�es. Il �xiste diff�rents statuts d'Audit d�crivant le statut du message transitant, par exemple un message a comme statut PENDING � sa g�n�ration, SENT � son envoi si un acquittement est attendu, SENT NO\_ACK � son envoi si aucun acquittement est attendu, ACKNOWLEDGED � son acquittement, ERROR en cas d'erreur . \\

Il est possible de param�trer des taches automatiques � �x�cuter lorsqu'un message transitant par Amex passe dans un certain statut d'audit. C'est le role du service de taches automatiques. Ces taches automatiques peuvent etre un envoi d'un mail, un appel � un web service distant ou le post d'un message sur une file JMS ou MQ.
Il �xiste des objets TaskHandler permettant de savoir comment cr�er le message � envoyer et � qui l'envoyer. Ces informations sont contenues dans la table TaskHandler et sont param�trables via des arbres RuleSolver. 


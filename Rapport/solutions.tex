\chapter{Les solutions techniques}

\section{Synthèse de l'existant}
Avant de vous décrire la partie technique de mon travail, je vais vous introduire les principales technologies et principaux logiciels que j'ai utilisé. \\

\subsection{Langage de développement}
L'application sur laquelle je travaille est développé en Java EE, soit Java Entreprise Edition. C'est la version de Java destinée aux applications des entreprises. 

\subsection{Les outils utilisés}

\subsection{Les termes techniques}

\section{Description du travail}

\subsection{Analyse de l'éxistant}
La première phase de mon travail a consisté en une analyse de la modélisation base de données des FlowDetails. Après avoir installé l'environnement de travail et tous les outils nécessaires, j'ai pu analyser le projet et plus particulièrement tout ce qui concernait les FlowDetails. \\

Ainsi, j'ai pu réaliser un diagramme entité relations représentant la modélisation base de données des FlowDetails. Voir annexe : Modélisation des FlowDetails existants. Cela m'a permis d'analyser tous les objets en relation avec les FlowDetails.  

\subsubsection{Les API utilisant les FlowDetails}
Dans la plateforme Amex, il y a différentes services  qui utilisent les flowdetails. Je vais décrire le fonctionnement de chacun d'eux. \\

Le service d'audit a pour but de tracer tous les messages transitant par Amex. L'audit se fait dans des tables en base de données. Il éxiste différents statuts d'Audit décrivant le statut du message transitant, par exemple un message a comme statut PENDING à sa génération, SENT à son envoi si un acquittement est attendu, SENT NO\_ACK à son envoi si aucun acquittement est attendu, ACKNOWLEDGED à son acquittement, ERROR en cas d'erreur . \\

Il est possible de paramétrer des taches automatiques à éxécuter lorsqu'un message transitant par Amex passe dans un certain statut d'audit. C'est le role du service de taches automatiques. Ces taches automatiques peuvent etre un envoi d'un mail, un appel à un web service distant ou le post d'un message sur une file JMS ou MQ.
Il éxiste des objets TaskHandler permettant de savoir comment créer le message à envoyer et à qui l'envoyer. Ces informations sont contenues dans la table TaskHandler et sont paramétrables via des arbres RuleSolver. 



\subsection{La nouvelle modélisation}

\subsection{Migration de l'existant}

\section{Protocole d'évaluation}

